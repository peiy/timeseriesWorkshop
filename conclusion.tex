%%In this paper, we study the problem of clustering
%%gamer's time series.  In particular, we solved two tasks: 
%%1) identifying gamers' underlying playing states using HMH, 
%%and 2) clustering gamers' based on their state transitions using
%%a Dirichlet clustering method. 
%%We experimented on one of the strategy game developed by Zynga Inc,
%%and the results reveal interesting results that can help improve the game design.

There are many ways to cluster time series and in previous sections we reported on some of those. The specific approach we took, to first identify latent states and then cluster the time series of the latent states, is motivated by the need to have game designers interact with the results.  There are mainly two kind of interactions we aimed at: first, the results have to be interpretable by the game designers, and second we need to give game designers a way to provide side information and influence the results.  We found that by using latent states, and ``naming'' the latent states using the properties of the distribution of the actions they generate, the game designers and product managers could get to actionable information from the clustering.  The use of the Dirichlet process approach also allows them to provide side information regarding which players should and should not be in the same clusters.  We are currently studying the best ways to quantify these interactions. 

