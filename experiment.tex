We apply our method to one of  Zynga's strategy game where the goal is to conquer all the battlefields
in a global map (players can play against other players or against the game itself). The players need to build/upgrade base resources with  weapons and troops, which in turn 
requires game points that can be obtained from winning battles.  Thus,  players need to 
tradeoff between building resources and conquering battlefields.

%We apply our method to one of Zynga Inc's mobile strategy game -- Empires and Allies, an all-new modern military strategy game that players can design their army, build their base and deploy their weapons and troops to conquer the battlefields. Thanks to the powerful backend system, the game tracks almost every action each player does in the game, which provide us a handful data set to investigate the sophisticated player types. We choose a data set which contains a daily active cohort's 10 consecutive days data from the first day they install the game. For each player and each day, we have 67 features including engagement features like session length, number of battles, game action features like deploy a tank, build a defence building and join an alliance. Then we use the feature selection algorithm to choose 5 out of 67 features to build the time series clustering model. They are PvP (People vs People) Battle , PvE (People vs Bot) Battle, Points gained/lost per battle, session length, Level up and is-Payer.  

We subsample players from  a dataset of 10 consecutive days data after installation. This results in 1719 players.
Each player is characterized with $67$ features at each day, and we select $5$ important 
features based on prior experience and feature selection methods: \texttt{PvP} (people vs people battle), 
\texttt{Pve} (people vs machine battle), \texttt{Points} (number of points gained), \texttt{Session} 
(number of session started), \texttt{LevelUp} (whether a player level up) and \texttt{isPayer} (whether
the player paid). We model \texttt{Points}  with Gaussian distribution, \texttt{Session} with
Poisson distribution, and the rest features with Bernoulli distribution.


%\begin{wrapfigure}{r}{0.4\textwidth}
%\centering
%\includegraphics[width=0.35\textwidth]{likeli} 
%\caption{Data likelihood using different number of playing states.}
%\label{fig:likeli}
%\end{wrapfigure}




{\bf Identifying Players' Latent States.} Our experiments yielded 3 states that were interpretable given the distributional characteristics of the features (see 
Table \ref{tab:states}): \textbf{Aggressive}, 
\textbf{Defensive}, and \textbf{Moderate}.  The \textbf{Aggressive} state captures 
the mode where the players focus on conquering battles, while the \textbf{Defensive} 
state describes the stage that they acquire the resources and build their base. The \textbf{Moderate} is a mixture
of the two. These states interestingly identified the design of the game explained previously,
namely, the players have to conquer battles and build resources intermittently in order to progress. 

\begin{table}
\parbox{.65\linewidth}{
\centering
\caption{Discovered Playing States}
\label{tab:states}
\centering
\scalebox{0.8}
{
\begin{tabular}{lccc} 
\bf Feature / State & \textbf{Aggressive}  & \textbf{Defensive}  &  \textbf{Moderate} 
\\ \hline \\
 Prob.\ \texttt{Pvp}     &  0.2472  & 0.0295 &  0.0947  \\
 Prob.\ \texttt{Pve}     &  0.2430  & 0.0581 &  0.1044 \\
 Mean \texttt{Points} &  88.10   & 1238.36 &  476.22 \\
 Mean \texttt{Session}&  4.58    &  34.19  & 12.95 \\
 Prob.\ \texttt{LevelUp} &  0.1664  & 0.0177  & 0.0553 \\
 Prob.\ \texttt{Pay}     &  0.0031  & 0.0956  & 0.0320 \\
\end{tabular}
}
}
\hfill
\parbox{.35\linewidth}{
\centering
\caption{Clustering results.}
\label{tab:clustering}
\begin{tabular}{ccc}
\bf Method & \bf \#Cluster & \bf DB
\\ \hline \\
{\bf Ours}     & {\bf 3}  &  {\bf 0.968} \\  
Kmeans  &  5  &  2.628 \\
GMM     &  4  &  1.803 
\end{tabular}
}
\end{table}

Figure \ref{fig:transitions} plots the transitions for all the players among the 10 days (decoded using Viterbi). 
The result shows that most of the users starts with the \textbf{Moderate} or \textbf{Defensive} state, 
and then gradually transition to the \textbf{Aggressive} state. 
This is  consistent with the initial game design as  it is difficult for players  to start with many battles due to  resource restrictions, but they ultimately need to become aggressive and conquer all the battlefields.

\begin{figure}[h]
\begin{minipage}[b]{0.3\linewidth}
    \centering
    \includegraphics[width=0.8\linewidth]{transitions} 
    \caption{State transitions of all the Players at the first 10 days of installing the game.}
    \label{fig:transitions}
\end{minipage}
\quad
\begin{minipage}[b]{0.65\linewidth}
    \centering
        \begin{tabular}{ccc}
         \includegraphics[width=0.33\textwidth]{cluster1} &
         \includegraphics[width=0.33\textwidth]{cluster2} &
         \includegraphics[width=0.33\textwidth]{cluster3} \\
         \multicolumn{3}{c}{ \includegraphics[width=0.75\textwidth]{legend} }
         \end{tabular}
     \caption{\label{fig:clusters} State transitions within three clusters found by our method.}
\end{minipage}
\end{figure}

%
%\begin{table}[t]
%\caption{Discovered Playing States}
%\label{tab:states}
%\centering
%\begin{tabular}{ccccccc}
%\bf State & \bf Pvp Pr. & \bf Pve Pr. & \bf Points Mean & \bf Session Mean & \bf LevelUp Pr. & \bf Pay Pr. 
%\\ \hline \\
%\texttt{Aggressive} & 0.2472  &  0.2430  &  88.10   &  4.58   & 0.1664  & 0.0031 \\  
%\texttt{Defensive}  & 0.0295  &  0.0581  &  1238.36 &  34.19  & 0.0177  & 0.0956 \\
%\texttt{Moderate}   & 0.0947  &  0.1044  &  476.22  &  12.95  & 0.0553  & 0.0320 
%\end{tabular}
%\end{table}


%\begin{figure}[t]
%\centering
%    \begin{tabular}{ccc}
%     \includegraphics[width=0.26\textwidth]{cluster1} &
%     \includegraphics[width=0.26\textwidth]{cluster2} &
%     \includegraphics[width=0.26\textwidth]{cluster3} \\
%     \multicolumn{3}{c}
%      { \includegraphics[width=0.45\textwidth]{legend} }
%    \end{tabular}
%     \caption{\label{fig:clusters} State transitions within the three clusters found by our method.
%              }
%\end{figure}
%
{\bf Clustering Results.} We compare the results of our clustering methods with two alternatives: Kmeans and Gaussian Mixture Model (GMM) by using a popular evaluation method based on Davis-Bouldin (DB) index \cite{dbindex} (we remind the readers that we don't have the ground truth). 
We tune the number of clusters for Kmeans using DB index, and report the one that has the best (the smallest) DB index value. 
Table \ref{tab:clustering} reports the clustering results of all methods, 
where our method outperforms the alternatives. We also plot the state transitions within each clusters  
and found that our method produces the most meaningful results. Here we show the within cluster transitions found by our
method in Figure \ref{fig:clusters}.  


