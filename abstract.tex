Game industry plays an important in improving our entertainment in digital life. 
One of the main tasks for the game companies is to develop user-specific features to 
improve their gaming experience. The key factor is to understand player's behaviors during their 
game-play. This paper  considers the following two tasks: 
Given user's time series behavior data, 1) what are the underlying playing states that 
controls user's behaviors at each time period, and 2) what are the natural groups of players based 
on their state transitions. We apply Hidden Markov Model (HMM) to identify gamers' 
hidden playing states, and use a Dirichlet clustering method to find player groups based on 
their state transitions. Our case study on one of the strategy game developed by Zynga Inc.\  
reveals interesting results that can help gamer designers to improve the game.

