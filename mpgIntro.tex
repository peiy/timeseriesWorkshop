The game industry is steadily gaining grounds in the competition for our digital time. Besides the traditional economic and business analysis reports, evident of this fact is the declared interest of companies like Amazon (purchased Twitch), Sony (playStation), Microsoft (XBOX), to name but a few.  Games are also an incredibly rich source of data about human behaviors ranging from social interactions to economical and rational decision making.

From the points of view of improving the player experience and also from the business aspects of retention, engagement  and payments, it is very important for game developers to understand  player behaviors and take appropriate actions accordingly. As players change their playing strategies and patterns during their involvement with the game (which can take years), and as these patterns are clearly not independent from previous playing behavior, the proper analysis of these patterns should be done by looking at the time series of the player actions. These time series are multivariate in nature, as players can take over hundreds of actions in some of these games, and these actions are further parameterized by real values.

In this paper, we explore an approach to characterizing player behaviors from the multivariate time series of their raw actions in the game based on 
1) inducing a time series of latent states that abstract their raw actions and 2) clustering these time series of latent states.  To induce the time series of latent states we rely on  Hidden Markov Model (HMM) \cite{hmm}, with the playing states as 
the hidden nodes and the player action statistics at each time period as the observed features.  We fit the HMM model and induce the latent  states by applying the Viterbi decoding algorithm.  We then find the groupings of players using an unsupervised clustering method based on modeling this grouping as a Bayesian version of a Dirichlet Process~\cite{}.

The first step provides game designers and product managers with visibility into what are the individual modes of play so they can examine how a mixture of game actions give raise to strategy and enable them to plan for changes.  The second step provides visibility into the aggregate behavior of players so that business decisions can be made regarding key metrics such as retention, engagement, and monetization. 

Of course, there exists previous work on clustering time series data with HMM models
\cite{Bicego2006,bicego2003,coviello2014}. However, most of the work we reviewed focuses on finding
out the final clustering result, while, as explained above we are also interested in studying
the latent states of each player at each time period and are interested in the interpretability of the final model. In the game domain, Men{\'e}ndez et al.~\cite{menendez2014} propose an approach to extract user profiles (based on 5 pre-defined metrics) from player time series data. That task is very different from ours
as we are interested in also identifying the player latent states. %and their methods don't apply. 


We test our approach on a strategy game developed by Zynga Inc. %, named {\it Empires and Allies}. 
Our method found three basic latent  states generating all the daily raw actions, and  the clustering results reveal three 
 groups that contain players with similar patterns of transition states and strategy changes over time.  These groupings enabled product managers to estimate correlations with key business metrics and what changes would induce changes in these metrics.  The understanding of player latent states provide evidence for hypothesis and A/B testing experiments to improve on player experience and their progress through the different levels of the game.
% this needs to be strengthen...
%%As we will argue these results greatly simplify the task of  understanding players' behaviors to improve the game design and other business metrics as monetization and engagement. This characterization enables the game designers and managers to track and identify on a weekly basis the changes in behavior and how it correlates to changes in design, reaction to incentives, and A/B testing.  


